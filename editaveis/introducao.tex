\chapter{Introdução}
Diante do crescimento que várias cidades atingem depois de certo grau de urbanização, o trânsito se torna um problema constante, com carros e motos disputando espaços limitados em vias, criando grandes congestionamentos e trazendo consigo um tempo de espera entre destinos pouco desejada entre usuários. Devido a esses fatores, bicicletas surgem como uma alternativa para trajetos de curta e média distância, pois conseguem aliar uma velocidade satisfatória com a possibilidade de se evitar aglomerações de veículos automotores, visto que geralmente, estas possuem espaços físicos destinados a ela, e mesmo em vias comuns a todos os veículos, podem trafegar em uma área mais específica. 

Ainda sim, a bicicleta apesar de suas vantagens, ainda possui uma vulnerabilidade considerável em termos de criminalidade e segurança de trânsito. Segundo a ANTP, é possível identificar que de maneira geral, os principais medos e fatores do não uso das bicicletas, estão relacionados ao medo de ser atropelado e atitude de motoristas quanto aos ciclistas, ou seja, a insegurança no trânsito e o medo de furtos e assaltos.  Além disso, o site Bicicletas Roubadas, por meio do Cadastro Nacional de Bicicletas Roubadas, uma das poucas fontes de dados relacionada a furtos e assaltos relacionados a bicicletas, visto que não existem dados oficiais sobre essa modalidade de crime, indica que já ocorreram em 2017, 119 ocorrências em todo o país.

A partir desse panorama geral apresentado, este relatório visa apresentar conceitualmente uma solução de engenharia visando permitir um maior conforto e segurança aos usuários de bicicletas através de um sistema de iluminação que permita a comunicação com o GPS do celular/smartphone do usuário, indicando a ele a rota que ele deve seguir de acordo com o desejado, porém sem a necessidade de se manter o aparelho exposto, visando diminuir a possibilidade de roubos, além de servir como um elemento identificador do ciclista em si no trânsito, devido à iluminação que servirá ao ciclista também podendo ser visualizada por outros motoristas ou pedestres que estejam dividindo o espaço de tráfego com ele. 

	\subsection{Descrição do Problema}
	A descrição do problema se encontra na Tabela \ref{table_descricao_problema}, onde é descrito o problema, os indivíduos afetados pelo problema, o impacto e uma possível solução.
\begin{table}[H]
\centering
\caption{Tabela de descrição do Problema}
\label{table_descricao_problema}
\begin{tabular}{|ll|}
\hline
\textbf{Problema}            & \begin{tabular}[c]{@{}l@{}}Necessidade de se utilizar o celular/smartphone\\ para visualizar o trajeto desejado, estando \\ vulnerável a assaltos e/ou não atenção correta na via\\ devido ao uso do equipamento;\end{tabular} \\ \hline
\textbf{Indivíduos Afetados} & \begin{tabular}[c]{@{}l@{}}Ciclistas e potenciais usuários devido ao\\  temor em relação ao problema;\end{tabular}                                                                                                             \\ \hline
\textbf{Impacto}             & \begin{tabular}[c]{@{}l@{}}Menor quantidade de pessoas utilizando\\ um meio de transporte ambientalmente \\ amigável e pressão destas nos outros meios\\ de transporte;\end{tabular}                                           \\ \hline
\textbf{Solução}             & \begin{tabular}[c]{@{}l@{}}Projeto de um sistema de iluminação que \\ se comunique com o GPS do aparelho do \\ usuário e consiga indicar a direção do trajeto \\ sem a necessidade do uso do aparelho;\end{tabular}            \\ \hline
\end{tabular}
\end{table}
	
	
	\subsection{Justificativa}
	Diante do exposto, relativo aos problemas urbanas de uso veículos de maior parte, e viabilidade do uso da bicicleta como alternativa, aliado aos dados obtidos da ANTP, o equipamento seria interessante por abarcar uma solução que atuaria tanto no aspecto de segurança contra roubos, evitando expor aparelhos que demonstrem um valor elevado para potenciais assaltantes, bem como para segurança contra acidentes, permitindo ao ciclista ser mais notado por  motoristas, outros ciclistas ou mesmo pedestres, além disso, ainda não existe nenhum equipamento semelhante no mercado que possua esse funcionamento atuando nessas duas frentes. 
